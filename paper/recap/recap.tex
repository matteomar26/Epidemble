\documentclass[a4paper, amsfonts, amssymb, amsmath, reprint, showkeys, nofootinbib, twoside, floatfix, pre,superscriptaddress]{revtex4-2}

\usepackage[utf8]{inputenc}
% \usepackage{geometry}
% \geometry{verbose,lmargin=5cm,rmargin=2cm}
\usepackage[left=23mm,right=13mm,top=35mm,columnsep=15pt]{geometry} 
\setcounter{secnumdepth}{3}
\usepackage{amsmath}
\usepackage{amssymb}
\usepackage{esint}
\usepackage[unicode=true, pdfusetitle, bookmarks=true, bookmarksnumbered=false, bookmarksopen=false, breaklinks=false, pdfborder={0 0 1}, backref=false, colorlinks=false]{hyperref}
%\usepackage[margin=0.5cm]{subcaption}
\usepackage{graphicx}
\graphicspath{ {./images/} }

\makeatletter
%%%%%%%%%%%%%%%%%%%%%%%%%%%%%% User specified LaTeX commands.
%\pdfoutput=1
%
% Document outline for use when preparing LaTeX manuscripts
% for Elsevier Major Reference Works
% To run with LaTeX $2\epsilon$
%
% MJR, October 2003
%

\usepackage{makeidx}\usepackage{amsfonts}

%%%%%%%%%%%%%%%%%%%%%%%%%%%%%%%%%%%%%%%%

\makeatother

\begin{document}
\title{Ensemble Study for Inference of Epidemic Trajectories}

\maketitle

\section{Ensemble Study for Inference of Epidemic Trajectories}
\subsection{Epidemic Inference}
{\it SI model on graphs.}
We consider the Susceptible-Infected (SI) model of spreading, defined over a graph $\mathcal{G}=(V,E)$.
At time $t$ a node $i\in V$ can be in two states represented by a variable $x_i^t\in\{S,I\}$.
At each time step, an infected node can infect each of its susceptible neighbors $\partial i$ with independent probabilities $\lambda_{ij}\in[0,1]$.
The dynamic is irreversible: a given node can only undergo the transition $S\to I$. 
Therefore the trajectory of an individual can be parameterized by its infection time $t_i$.
We assume that a subset of the nodes initiate with an infection time $t_i=0$, i.e. $x_i^0=I$.
A realization of the SI process can be univocally expressed in terms of the independent transmission delays $s_{ij}\in\{0, 1, \dots, \infty\}$, following a geometrical distribution $w_{ij}(s)=\lambda_{ij}(1-\lambda_{ij})^s$.
Once the initial condition $\{x_i^0\}_{i\in V}$ and the set of transmission delays $\{s_{ij}\}_{(ij)\in E}$ is fixed, the infection times can be uniquely determined from the set of equations
$$
	t_i=\delta_{x_i^0,S}(\min_{j\in\partial i}\{t_j+s_{ji}\}+1)
$$
We assume that each individual has a probability $\gamma$ to be infected at time $t=0$, and we assume for simplicity that the transmission probabilities are site-independent: $\lambda_{ij}=\lambda$ for all $(ij)\in E$.
The distribution of infection times conditioned on the realization of delays and on the initial condition can be written:
\begin{align*}
	P(\{t_i\}|\{x_i^0\},\{s_{ij}\})&=\prod_{i\in V}	\psi^*(t_i, t_{\partial i}, x_i^0, \{s_{ji}\}_{j\in\partial i}) 
\end{align*}
where $\psi^*$ enforces the above constraint on the infection times:
\begin{align}
	\label{eq:constraint_infection_times}
	\psi^*=\mathbb{I}[t_i=\delta_{x_i^0,S}(\min_{j\in\partial i}\{t_j+s_{ji}\}+1)]
\end{align}
with $\mathbb{I}[A]$ the indicator function of the event $A$.
Once averaged over the transmission delays and over the initial condition, we obtain the following distribution of times:
\begin{align}
	\label{eq:forward_averaged}
	P(\{t_i\})&=\prod_{i\in V}\psi(t_i, t_{\partial i})
\end{align}
where:
\begin{align*}
	\psi=\sum_{x_i^0}\gamma(x_i^0)\sum_{\{s_{ji}\}_{j\in\partial i}}\prod_{j\in\partial i}w(s_{ji})\psi^*(t_i, t_{\partial i}, x_i^0, \{s_{ji}\}_{j\in\partial i})
\end{align*}

{\it Inferring individual's trajectories from partial observations.}
In the inference problem we assume that some information $\mathcal{O}=\{o_i\}_{i\in\mathcal{S}}$ on the trajectory of a subset $\mathcal{S}$ of individuals is given by the result of medical tests. 
Most of the time we will take $o_i=x_i^T$, i.e. we observe the state of an individual at time $t=T$.
The probability of observations $P(\mathcal{O}|\{t_i\})$ factorizes over the individuals:
\begin{align}
\label{eq:observations}
P(\mathcal{O}|\{t_i\})=\prod_{i\in\mathcal{S}}\rho(x_i^T|t_i) \ .
\end{align}
Using Bayes rule, the posterior probability of infection times is:
\begin{align}
\label{eq:posterior}
	P(\{t_i\}|\mathcal{O}) = \frac{P(\{t_i\})P(\mathcal{O}|\{t_i\})}{P(\mathcal{O})}
\end{align}
with $P(\{t_i\})$ given in~(\ref{eq:forward_averaged}).
In the simplest case, the state of each individual at time $t=T$ is perfectly known: $\mathcal{O}=\{x_i^T\}_{i\in V}$, and:
\begin{align*}
	\rho(x_i^T|t_i)=\mathbb{I}[x_i^t=r(t_i) ] \\
	{\rm with} \quad r(t)=\begin{cases}
		I & \text{if $t_i\leq T$}\\
		S & \text{if $t_i>T$}
\end{cases} \ .	
\end{align*}
One can also introduce some uncertainty in the result of medical tests, with probability $p$:
\begin{align}
	\label{eq:prob_falserate}
	\rho(x_i^T|t_i)=(1-p)\mathbb{I}[x_i^T= r(t_i)] + p\mathbb{I}[x_i^T= 1-r(t_i)] 
\end{align}
(here for simplicity we take the same value for FNR and FPR: $p_{\rm FNR}=p_{\rm FPR}=p$, but we could generalize to different FNR and FPR values.)
In the Bayes optimal setting, the parameters $(\lambda,\gamma, p)$ of the true trajectory are known, this means that the parameters ($\gamma, \lambda, p$) used in the posterior probability~(\ref{eq:posterior}) are the same than the true parameters used to generate the observations.
However in real cases, the value of the parameters is not known, and we denote by ($\lambda^*, \gamma^*, p^*$) (resp. $\lambda^I, \gamma^I, p^I$) the parameters used to generate the observations (resp. to infer the infection times).

\subsection{A graphical model for the joint distribution over planted and inferred trajectories}
Our objective is to estimate how close is the time $t_i$ inferred from the posterior distribution given the observations $\mathcal{O}$ from the true infection time (that we denote $\tau_i$). 
We consider the joint distribution over the true (or planted) times $\{\tau_i\}_{i\in V}$ and the inferred times $\{t_i\}_{i\in V}$:
\begin{align}
\label{eq:joint}
\begin{aligned}
	P(\{t_i\}, \{\tau_i\}) &= P(\{\tau_i\})\sum_{\mathcal{O}}P(\mathcal{O}|\{\tau_i\})P(\{t_i\}|\mathcal{O},\{\tau_i\})\\
	&= P(\{\tau_i\})\sum_{\mathcal{O}}P(\mathcal{O}|\{\tau_i\})P(\{t_i\}|\mathcal{O})\\
	&=P(\{\tau_i\})P(\{t_i\})\sum_{\mathcal{O}}\frac{P(\mathcal{O}|\{\tau_i\})P(\mathcal{O}|\{t_i\})}{P(\mathcal{O})}
\end{aligned}
\end{align}
where in the second line we used $P(\{t_i\}|\mathcal{O},\{\tau_i\})=P(\{t_i\}|\mathcal{O})$, i.e. the probability law of $\{t_i\}$ conditioned on the observations $\{O\}$ and on the planted times $\{\tau_i\}$ depends only on the observations. In the third line we used the Bayes law~(\ref{eq:posterior}).
In the Bayes optimal setting, i.e. when $(\lambda^*, \gamma^*, p^*)=(\lambda^I, \gamma^I, p^I)$, the joint probability $P(\{t_i\}, \{\tau_i\})$ is invariant under the permutation of its two arguments $\{t_i\}, \{\tau_i\}$.

The probability distribution~(\ref{eq:joint}) cannot a priori be written as a graphical model because of the sum over the observations $\mathcal{O}$ and of the denominator $P(\mathcal{O})=\sum_{\{t_i\}}P(\{t_i\})P(\mathcal{O}|\{t_i\})$. 
We instead consider the joint probability distribution {\it conditioned} on the realization of the true initial condition $\{x_i^0\}$, on the delays $\{s_{ij}\}$, and on the the realization of the noise in the observations. For the last one we introduce binary variables $c_i$, with $c_i=0$ when the observation is not corrupted ($x_i^T=r(\tau_i)$), and $c_i=1$ when the observation is corrupted: $x_i^T=\overline{r(\tau_i)}$ (with $\overline{r}$ the negation of $r$: if $r=I$ then $\overline{r}=S$). In this way, each $c_i$ is a Bernoulli variable of parameter $p$.
We denote $\mathcal{D}=\{\{x_i^0, c_i\}_{i\in V}, \{s_{ij}\}_{(ij)\in E}\}$ a realization of the disorder.
The joint probability of the planted times $\{\tau_i\}$, of the observations $\mathcal{O}=\{x_i^T\}$ and of the inferred times conditioned on the disorder is:
\begin{align*}
	P(\{t_i\},\mathcal{O},&\{\tau_i\}|\mathcal{D}) = P(\{\tau_i\}|\mathcal{D})P(\mathcal{O}, \{t_i\}|\mathcal{D},\{\tau_i\}) \\
	&=P(\{\tau_i\}|\mathcal{D})P(\mathcal{O}|\mathcal{D},\{\tau_i\})P(\{t_i\}|\mathcal{D},\{\tau_i\}, \mathcal{O})\\
	&=P(\{\tau_i\}|\mathcal{D})P(\mathcal{O}|\mathcal{D},\{\tau_i\})P(\{t_i\}|\mathcal{O})
\end{align*}
where in the last line we have again noted that the posterior distribution on the inferred times $\{t_i\}$ depends only on the observations: $P(\{t_i\}|\mathcal{D},\{\tau_i\}, \mathcal{O})=P(\{t_i\}|\mathcal{O})$.
The first term in the product is:
$$
	P(\{\tau_i\}|\mathcal{D})=\prod_{i\in V}\psi^*(\tau_i, \underline{\tau}_i;x_i^0, \{s_{ji}\}_{j\in\partial i})
$$
with $\psi^*$ given in~(\ref{eq:constraint_infection_times}).
The second term in the product is the probability of having observation $\mathcal{O}=\{x_i^T\}$ given the planted times $\{\tau_i\}$ and de disorder $\mathcal{D}$. Each $x_i^T$ is a deterministic function of $\tau_i$ and of the corruption variable $c_i$:
$$
	P(\mathcal{O}|\mathcal{D},\{\tau_i\}) = \prod_{i\in V}\mathbb{I}[x_i^T=c_i\overline{r(\tau_i)}+(1-c_i)r(\tau_i)]
$$
The third term is expressed using Baye's law:
$$
	P(\{t_i\}|\mathcal{O})=\frac{P(\{t_i\})P(\mathcal{O}|\{t_i\})}{P(\{O\})}
$$
with $P(\{t_i\})$ given in~(\ref{eq:forward_averaged}) and $P(\mathcal{O}|\{t_i\})$ given in~(\ref{eq:observations}) (with $\rho(x_i^T|\tau_i)$ given in~(\ref{eq:prob_falserate})).
Finally, the denominator $$P(\mathcal{O})=\sum_{\{t_i\}}P(\{t_i\})P(\mathcal{O}|\{t_i\})$$ can be seen as a complicated function of the observations $\mathcal{O}$, but since we have fixed the disorder $\mathcal{D}=\{\{x_i^0, c_i\}_{i\in V}, \{s_{ij}\}_{(ij)\in E}\}$, the observations are a deterministic function of the disorder:
$x_i^T=c_i\overline{r(\tau_i)}+(1-c_i)r(\tau_i)$, and $\tau_i$ is itself a function of the initial condition $\{x_i^0\}$ and of the delays $\{s_{ij}\}$. So we can re-write it as a normalization that depends only on the disorder:
$$
	P(\mathcal{O})=Z(\mathcal{D})
$$
We finally obtain a joint-probability on $\{\tau_i\}$, $\mathcal{O}$ and $\{t_i\}$ that is factorized (graphical model):
\begin{align*}
	P(\{t_i\}&,\mathcal{O},\{\tau_i\}|\mathcal{D}) = \frac{1}{Z(\mathcal{D})}\prod_{i\in V}\psi^*(\tau_i, \underline{\tau}_{\partial_i};x_i^0,\{s_{ji}\}_{j\in\partial i}) \\
	&\times \psi(t_i,\underline{t}_{\partial i})\mathbb{I}[x_i^T=c_i\overline{r(\tau_i)}+(1-c_i)r(\tau_i)] \rho(x_i^T|t_i)
\end{align*}
Summing over the observations $\mathcal{O}=\{x_i^T\}$ is harmless since only one configuration $\{x_i^T\}_{i\in V}$ is accepted due to the indicator function above ($x_i^T$ is fixed by the disorder).
We obtain the joint probability distribution of planted and inferred times $\{\tau_i\}, \{t_i\}$ conditioned on the disorder:
\begin{align}
\begin{aligned}
	P(\{\tau_i\}, \{t_i\}|\mathcal{D})=\frac{1}{Z(\mathcal{D})}\prod_{i\in V}&\psi^*(\tau_i, \underline{\tau}_{\partial_i};x_i^0,\{s_{ji}\}_{j\in\partial i}) \\
&\times \psi(t_i,\underline{t}_{\partial i})\xi(\tau_i, t_i;c_i)
\end{aligned}
\end{align}
with:
\begin{align}
\begin{aligned}
	\xi(\tau_i, t_i;c_i) &= \rho(x_i^T|t_i) \\
	{\rm where} \quad x_i^T&=c_i\overline{r(\tau_i)}+(1-c_i)r(\tau_i)
\end{aligned}
\end{align}
with $\rho(x|t)$ given in~(\ref{eq:prob_falserate}).

{\it The case of perfect observations.}
In that case the probability of error is zero: $p=0$ so the corrupted variables are always $c_i=0$ (no corruption), and $\rho(x_i^T|t_i)=\mathbb{I}[x_i^T=r(t_i)]$. The coupling term $\xi(\tau_i, t_i;c_i)$ between inferred and planted times in the joint probability becomes:
\begin{align*}
	\xi(\tau_i, t_i)=\mathbb{I}[r(\tau_i)=r(t_i)] \\
	{\rm where} \quad r(t)=\begin{cases}
	I & \text{if $t_i\leq T$}\\
	S & \text{if $t_i>T$}
	\end{cases} \ .
\end{align*}

\subsection{Belief-Propagation equations for the joint-probability}
We introduce as auxiliary variables the copied times $\tau_i^{(j)}=\tau_i$, and $t_i^{(j)}=t_i$ for all $j\in\partial i$
\subsection{Replica-Symmetric Equations}
\section{Results}
\subsection{Check against simulations}
\subsection{Varying epidemic's parameters}
\subsection{Varying the false positive/negative rates}


\bibliography{draft}

\end{document}
